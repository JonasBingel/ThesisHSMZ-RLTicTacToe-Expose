% Generelle QoL Anweisungen ------------------------------------------------------------

% Befehle zu Abkuerzungen --------------------------------------------------------------
% Die Anweisung \, erzeugt einen kurzen Abstand und wird bei Abkuerzungen oder zwischen Zahlen und Masseinheiten verwendet
\newcommand{\bs}{$\backslash$\xspace}
\newcommand{\ua}{\mbox{u.\,a.}\xspace}
\newcommand{\oa}{\mbox{o.\,a.}\xspace}
\newcommand{\bspw}{bspw.\xspace}
\newcommand{\bzw}{bzw.\xspace}
\newcommand{\ca}{ca.\xspace}
\newcommand{\dahe}{\mbox{d.\,h.}\xspace}
\newcommand{\etc}{etc.\xspace}
\newcommand{\eur}[1]{\mbox{#1\,\texteuro}\xspace}
\newcommand{\evtl}{evtl.\xspace}
\newcommand{\ggfs}{ggfs.\xspace}
\newcommand{\Ggfs}{Ggfs.\xspace}
\newcommand{\gqq}[1]{\glqq{}#1\grqq{}}
\newcommand{\idR}{i.d.R.\xspace}
\newcommand{\inkl}{inkl.\xspace}
\newcommand{\insb}{insb.\xspace}
\newcommand{\usw}{usw.\xspace}
\newcommand{\Vgl}{Vgl.\xspace}
\newcommand{\sogn}{sogn.\xspace}
\newcommand{\zB}{\mbox{z.\,B.}\xspace}

% Custom Anweisungen ------------------------------------------------------------------
\newcommand\includepdfWithChapter[4]{%
\includepdf[pages={#1}, pagecommand={\chapter{#3}}]{#4}
\includepdf[pages={#2}]{#4}
}

